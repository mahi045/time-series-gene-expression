\subsection{Datasets}
We use three real datasets in our experiments and each dataset contains different number of genes and their associated time values of equal or unequal length.

\paragraph*{Yeast} The first dataset considered in our tests, denoted as \textbf{Yeast} and originally described in \cite{first_dataset}, contains the genome characterization of the mRNA transcript levels during the cell cycle of the yeast \textit{Saccharomyces cerevisiae}. Gene expression levels were gathered at regular intervals during the cell cycle. In particular, measurements were performed at $17$ time points with an interval of $10$ minutes between each pair of recorded values. The gene expression time series of this dataset are known to be associated to $5$ different phases, namely \texttt{Early G1}, \texttt{Late G1}, \texttt{S}, \texttt{G2} and \texttt{M} which represent the class values in their setting.
\paragraph*{$\mathbf{MS-rIFN\beta}$}
The second dataset, indicated as $\mathbf{MS-rIFN\beta}$ and first analyzed in \cite{second_dataset}, contains gene expression profiles of patients suffering from relapsing-remitting multiple sclerosis (MS), who are classified as either good or poor responders to recombinant human interferon beta (rIFN$\beta$). The dataset is composed by the expression profiles of $70$ genes isolated from each patient at $7$ time points: before the administration of the first dose of the drug ($t= 0$), every $3$ months ($t=1, 2, 3, 4$) and every $6$ months ($t=5,6$) in the first and second year of the therapy, respectively. For a few patients entire profile measurements are missing at $1$ or $2$ time points. From the complete MS-rIFN$\beta$ dataset we retained only $12$ genes whose expression profiles at $t = 0$ have shown to accurately predict the response to MS-rIFN$\beta$, as described in [6]. Furthermore, for each possible number of time points from $2$ to $7$ we extracted the corresponding gene expression time series, in order to obtain $6$ different datasets.
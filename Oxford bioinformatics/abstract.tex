\textbf{Motivation:}
Time series microarray experiments are widely used to study different dynamical biological processes. Monitoring the change in gene expression patterns over time provides the
distinct possibility and opportunity of studying the mechanistic characteristics in various cellular responses. 
%Therefore a lot of works has been done focusing on microarray based gene expression data. 
Howover, most of the works on time based gene expression  focus on clustering the relative genes. But like any other domain, microarray based gene expression has also some common properties as time series. Different domains (Financial stock prediction or weather forecast) use time series data for forecasting. Moreover, much studies have been done to find anomalous data points in time series in order to isolate noise from trend and seasonality. Therefore, a study of these approaches would to interesting to evaluate and invent new feature in time series gene expression data.

\textbf{Result:}
In this paper, we try to apply several time series gene expression data for anomaly detection and forecasting. In order to detect  anomaly for unlabelled data, we first try to fit the whole time series of a gene expression using some common statistical methods to identify anomalous points. Then we perform different unsupervised and supervised machine learning based approach to compare their performance. For labelled dataset we propose two deep learning based techniques for anomaly classification and compare their performance with machine learning based method.
Moreover, we apply deep learning based Neural Network and LSTM methods to forecast from previous values in gene expression and compare their performance with popular statistical method like ARIMA and Holtz Winter. This paper shows strength and weakness of different approach in analyzing time series based gene expression.
